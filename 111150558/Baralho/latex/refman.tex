\documentclass[a4paper]{book}
\usepackage{a4wide}
\usepackage{makeidx}
\usepackage{graphicx}
\usepackage{multicol}
\usepackage{float}
\usepackage{listings}
\usepackage{color}
\usepackage{textcomp}
\usepackage{alltt}
\usepackage{times}
\usepackage{ifpdf}
\ifpdf
\usepackage[pdftex,
            pagebackref=true,
            colorlinks=true,
            linkcolor=blue,
            unicode
           ]{hyperref}
\else
\usepackage[ps2pdf,
            pagebackref=true,
            colorlinks=true,
            linkcolor=blue,
            unicode
           ]{hyperref}
\usepackage{pspicture}
\fi
\usepackage[utf8]{inputenc}
\usepackage[brazil]{babel}
\usepackage{doxygen}
\lstset{language=C++,inputencoding=utf8,basicstyle=\footnotesize,breaklines=true,breakatwhitespace=true,tabsize=8,numbers=left }
\makeindex
\setcounter{tocdepth}{3}
\renewcommand{\footrulewidth}{0.4pt}
\begin{document}
\hypersetup{pageanchor=false}
\begin{titlepage}
\vspace*{7cm}
\begin{center}
{\Large Baralho }\\
\vspace*{1cm}
{\large Gerado por Doxygen 1.7.1}\\
\vspace*{0.5cm}
{\small Quinta, 10 de Maio de 2012 20:58:55}\\
\end{center}
\end{titlepage}
\clearemptydoublepage
\pagenumbering{roman}
\tableofcontents
\clearemptydoublepage
\pagenumbering{arabic}
\hypersetup{pageanchor=true}
\chapter{Índice das Estruturas de Dados}
\section{Estruturas de Dados}
Aqui estão as estruturas de dados, uniões e suas respectivas descrições\-:\begin{DoxyCompactList}
\item\contentsline{section}{\hyperlink{structbaralho}{baralho} }{\pageref{structbaralho}}{}
\end{DoxyCompactList}

\chapter{Índice dos Arquivos}
\section{Lista de Arquivos}
Esta é a lista de todos os arquivos e suas respectivas descrições:\begin{DoxyCompactList}
\item\contentsline{section}{\hyperlink{corta_8c}{corta.c} }{\pageref{corta_8c}}{}
\item\contentsline{section}{\hyperlink{corta_8h}{corta.h} }{\pageref{corta_8h}}{}
\item\contentsline{section}{\hyperlink{embaralhar_8c}{embaralhar.c} }{\pageref{embaralhar_8c}}{}
\item\contentsline{section}{\hyperlink{embaralhar_8h}{embaralhar.h} }{\pageref{embaralhar_8h}}{}
\item\contentsline{section}{\hyperlink{retira_8c}{retira.c} }{\pageref{retira_8c}}{}
\item\contentsline{section}{\hyperlink{retira_8h}{retira.h} }{\pageref{retira_8h}}{}
\end{DoxyCompactList}

\chapter{Estruturas}
\hypertarget{structnodo}{
\section{Referência da Estrutura nodo}
\label{structnodo}\index{nodo@{nodo}}
}


{\ttfamily \#include $<$embaralhar.h$>$}

\subsection*{Campos de Dados}
\begin{DoxyCompactItemize}
\item 
int \hyperlink{structnodo_a0d153279003388933e082f40da4a3702}{carta}
\begin{DoxyCompactList}\small\item\em armazena uma carta (seu respectivo número). \item\end{DoxyCompactList}\item 
char \hyperlink{structnodo_aba763958bee1abb2961d0d2e82f8cdfb}{naipe}
\begin{DoxyCompactList}\small\item\em armazena um nipe. \item\end{DoxyCompactList}\item 
struct \hyperlink{structnodo}{nodo} $\ast$ \hyperlink{structnodo_a486ad5c9b955ff42eaf8fbf3330e4c75}{prox}
\begin{DoxyCompactList}\small\item\em apontador para o próximo nó. \item\end{DoxyCompactList}\end{DoxyCompactItemize}


\subsection{Descrição Detalhada}
This file is part of Baralho.

Copyright 2012 Cristiano Ribeiro $<$\href{mailto:cristiano.daitx@gmail.com}{\tt cristiano.daitx@gmail.com}$>$

Copyright 2012 Evair Severo $<$\href{mailto:evairsevero@gmail.com}{\tt evairsevero@gmail.com}$>$

This library is free software; you can redistribute it and/or

modify it under the terms of the GNU Lesser General Public

License as published by the Free Software Foundation; either

version 2.1 of the License, or (at your option) any later version.

This library is distributed in the hope that it will be useful,

but WITHOUT ANY WARRANTY; without even the implied warranty of

MERCHANTABILITY or FITNESS FOR A PARTICULAR PURPOSE. See the GNU

Lesser General Public License for more details.

You should have received a copy of the GNU Lesser General Public

License along with this library. If not, see $<$\href{http://www.gnu.org/licenses/}{\tt http://www.gnu.org/licenses/}$>$.

Estrutura que representa uma carta. Armazena a carta e seu respectivo naipe. Sendo as cartas representadas de 1 a 13. Logo as cartas A, J, Q e K são representadas, pelos números 1, 11, 12 e 13 respectivamente. As demais cartas seguem o padrão que conhecemos. Os nipes são representados por um caractere. Paus, copas, espadas e ouro são representados por P, C, E e O respectivamente. 

\subsection{Campos}
\hypertarget{structnodo_a0d153279003388933e082f40da4a3702}{
\index{nodo@{nodo}!carta@{carta}}
\index{carta@{carta}!nodo@{nodo}}
\subsubsection[{carta}]{\setlength{\rightskip}{0pt plus 5cm}int {\bf carta}}}
\label{structnodo_a0d153279003388933e082f40da4a3702}


armazena uma carta (seu respectivo número). 

\hypertarget{structnodo_aba763958bee1abb2961d0d2e82f8cdfb}{
\index{nodo@{nodo}!naipe@{naipe}}
\index{naipe@{naipe}!nodo@{nodo}}
\subsubsection[{naipe}]{\setlength{\rightskip}{0pt plus 5cm}char {\bf naipe}}}
\label{structnodo_aba763958bee1abb2961d0d2e82f8cdfb}


armazena um nipe. 

\hypertarget{structnodo_a486ad5c9b955ff42eaf8fbf3330e4c75}{
\index{nodo@{nodo}!prox@{prox}}
\index{prox@{prox}!nodo@{nodo}}
\subsubsection[{prox}]{\setlength{\rightskip}{0pt plus 5cm}struct {\bf nodo}$\ast$ {\bf prox}}}
\label{structnodo_a486ad5c9b955ff42eaf8fbf3330e4c75}


apontador para o próximo nó. 



A documentação para esta estrutura foi gerada a partir do seguinte arquivo:\begin{DoxyCompactItemize}
\item 
\hyperlink{embaralhar_8h}{embaralhar.h}\end{DoxyCompactItemize}

\chapter{Arquivos}
\hypertarget{corta_8c}{
\section{Referência do Arquivo corta.c}
\label{corta_8c}\index{corta.c@{corta.c}}
}
{\ttfamily \#include $<$stdio.h$>$}\par
{\ttfamily \#include $<$stdlib.h$>$}\par
{\ttfamily \#include $<$malloc.h$>$}\par
{\ttfamily \#include $<$time.h$>$}\par
{\ttfamily \#include $<$string.h$>$}\par
{\ttfamily \#include \char`\"{}embaralhar.h\char`\"{}}\par
{\ttfamily \#include \char`\"{}retira.h\char`\"{}}\par
{\ttfamily \#include \char`\"{}corta.h\char`\"{}}\par
\subsection*{Funções}
\begin{DoxyCompactItemize}
\item 
\hyperlink{structnodo}{Baralho} \hyperlink{corta_8c_ad6c1ede7439c49b9380e26032060b3cc}{cortar} (\hyperlink{structnodo}{Baralho} B, int pos)
\item 
void \hyperlink{corta_8c_a2a27d98e3838f3a5282d83ca3ed54f2d}{insere\_\-aux} (\hyperlink{structnodo}{Baralho} $\ast$A, int carta, char naipe)
\end{DoxyCompactItemize}


\subsection{Funções}
\hypertarget{corta_8c_ad6c1ede7439c49b9380e26032060b3cc}{
\index{corta.c@{corta.c}!cortar@{cortar}}
\index{cortar@{cortar}!corta.c@{corta.c}}
\subsubsection[{cortar}]{\setlength{\rightskip}{0pt plus 5cm}{\bf Baralho} cortar (
\begin{DoxyParamCaption}
\item[{{\bf Baralho}}]{ B, }
\item[{int}]{ pos}
\end{DoxyParamCaption}
)}}
\label{corta_8c_ad6c1ede7439c49b9380e26032060b3cc}
Corta um baralho em uma determinada posição. 
\begin{DoxyParams}{Parâmetros}
\item[{\em B}]-\/ Baralho a ser cortado. \item[{\em pos}]-\/ posição em que o baralho vai ser cortado. \end{DoxyParams}
\hypertarget{corta_8c_a2a27d98e3838f3a5282d83ca3ed54f2d}{
\index{corta.c@{corta.c}!insere\_\-aux@{insere\_\-aux}}
\index{insere\_\-aux@{insere\_\-aux}!corta.c@{corta.c}}
\subsubsection[{insere\_\-aux}]{\setlength{\rightskip}{0pt plus 5cm}void insere\_\-aux (
\begin{DoxyParamCaption}
\item[{{\bf Baralho} $\ast$}]{ A, }
\item[{int}]{ carta, }
\item[{char}]{ naipe}
\end{DoxyParamCaption}
)}}
\label{corta_8c_a2a27d98e3838f3a5282d83ca3ed54f2d}
Insere cartas em um baralho auxiliar. 
\begin{DoxyParams}{Parâmetros}
\item[{\em A}]-\/ Baralho a ser inserida a carta \item[{\em carta}]-\/ Numero da carta que vai ser inserida. \item[{\em carta}]-\/ Naipe da carta. \end{DoxyParams}

\hypertarget{corta_8h}{
\section{Referência do Arquivo corta.h}
\label{corta_8h}\index{corta.h@{corta.h}}
}
\subsection*{Funções}
\begin{DoxyCompactItemize}
\item 
\hyperlink{structnodo}{Baralho} \hyperlink{corta_8h_ad6c1ede7439c49b9380e26032060b3cc}{cortar} (\hyperlink{structnodo}{Baralho} B, int pos)
\item 
void \hyperlink{corta_8h_a2a27d98e3838f3a5282d83ca3ed54f2d}{insere\_\-aux} (\hyperlink{structnodo}{Baralho} $\ast$A, int carta, char naipe)
\end{DoxyCompactItemize}


\subsection{Funções}
\hypertarget{corta_8h_ad6c1ede7439c49b9380e26032060b3cc}{
\index{corta.h@{corta.h}!cortar@{cortar}}
\index{cortar@{cortar}!corta.h@{corta.h}}
\subsubsection[{cortar}]{\setlength{\rightskip}{0pt plus 5cm}{\bf Baralho} cortar (
\begin{DoxyParamCaption}
\item[{{\bf Baralho}}]{ B, }
\item[{int}]{ pos}
\end{DoxyParamCaption}
)}}
\label{corta_8h_ad6c1ede7439c49b9380e26032060b3cc}
\begin{DoxyAuthor}{Autores}
Cristiano Ribeiro, Evair Severo \href{mailto:cristiano.daitx@gmail.com}{\tt cristiano.daitx@gmail.com}, \href{mailto:evairsevero@gmail.com}{\tt evairsevero@gmail.com}
\end{DoxyAuthor}
Corta um baralho em uma determinada posição. 
\begin{DoxyParams}{Parâmetros}
\item[{\em B}]-\/ Baralho a ser cortado. \item[{\em pos}]-\/ posição em que o baralho vai ser cortado. \end{DoxyParams}
\hypertarget{corta_8h_a2a27d98e3838f3a5282d83ca3ed54f2d}{
\index{corta.h@{corta.h}!insere\_\-aux@{insere\_\-aux}}
\index{insere\_\-aux@{insere\_\-aux}!corta.h@{corta.h}}
\subsubsection[{insere\_\-aux}]{\setlength{\rightskip}{0pt plus 5cm}void insere\_\-aux (
\begin{DoxyParamCaption}
\item[{{\bf Baralho} $\ast$}]{ A, }
\item[{int}]{ carta, }
\item[{char}]{ naipe}
\end{DoxyParamCaption}
)}}
\label{corta_8h_a2a27d98e3838f3a5282d83ca3ed54f2d}
Função interna. Utilizada apenas por outras funções. Insere cartas em um baralho auxiliar. 
\begin{DoxyParams}{Parâmetros}
\item[{\em A}]-\/ Baralho a ser inserida a carta \item[{\em carta}]-\/ Numero da carta que vai ser inserida. \item[{\em carta}]-\/ Naipe da carta. \end{DoxyParams}

\hypertarget{descarte_8c}{
\section{Referência do Arquivo descarte.c}
\label{descarte_8c}\index{descarte.c@{descarte.c}}
}
{\ttfamily \#include $<$stdio.h$>$}\par
{\ttfamily \#include $<$stdlib.h$>$}\par
{\ttfamily \#include $<$malloc.h$>$}\par
{\ttfamily \#include $<$time.h$>$}\par
{\ttfamily \#include $<$string.h$>$}\par
{\ttfamily \#include \char`\"{}embaralhar.h\char`\"{}}\par
{\ttfamily \#include \char`\"{}retira.h\char`\"{}}\par
{\ttfamily \#include \char`\"{}corta.h\char`\"{}}\par
{\ttfamily \#include \char`\"{}descarte.h\char`\"{}}\par
\subsection*{Funções}
\begin{DoxyCompactItemize}
\item 
\hyperlink{structnodo}{Baralho} \hyperlink{descarte_8c_a7639a5e1a5d758731fe19ad2a258621f}{insereCartaDescarte} (\hyperlink{structnodo}{Baralho} D, int carta, char naipe)
\item 
\hyperlink{structnodo}{Baralho} \hyperlink{descarte_8c_afb7bcec544cd6814429992faa58a4d35}{atualizaDescarte} (\hyperlink{structnodo}{Baralho} D)
\item 
\hyperlink{structnodo}{lista} \hyperlink{descarte_8c_ae868e6817465c707db38fee6a1ac0ad8}{mostraDescarte} (\hyperlink{structnodo}{Baralho} D, int pos)
\item 
\hyperlink{structnodo}{lista} \hyperlink{descarte_8c_a1145f049a5b9cbb1b61842eb0a3948ee}{retiraDescarte} (\hyperlink{structnodo}{Baralho} $\ast$D, int pos)
\end{DoxyCompactItemize}


\subsection{Funções}
\hypertarget{descarte_8c_afb7bcec544cd6814429992faa58a4d35}{
\index{descarte.c@{descarte.c}!atualizaDescarte@{atualizaDescarte}}
\index{atualizaDescarte@{atualizaDescarte}!descarte.c@{descarte.c}}
\subsubsection[{atualizaDescarte}]{\setlength{\rightskip}{0pt plus 5cm}{\bf Baralho} atualizaDescarte (
\begin{DoxyParamCaption}
\item[{{\bf Baralho}}]{ D}
\end{DoxyParamCaption}
)}}
\label{descarte_8c_afb7bcec544cd6814429992faa58a4d35}
Insere uma determinada carta no monte de descarte. 
\begin{DoxyParams}{Parâmetros}
\item[{\em D}]-\/ Baralho de descarte onde vai ser inserida a carta. \end{DoxyParams}
\hypertarget{descarte_8c_a7639a5e1a5d758731fe19ad2a258621f}{
\index{descarte.c@{descarte.c}!insereCartaDescarte@{insereCartaDescarte}}
\index{insereCartaDescarte@{insereCartaDescarte}!descarte.c@{descarte.c}}
\subsubsection[{insereCartaDescarte}]{\setlength{\rightskip}{0pt plus 5cm}{\bf Baralho} insereCartaDescarte (
\begin{DoxyParamCaption}
\item[{{\bf Baralho}}]{ D, }
\item[{int}]{ carta, }
\item[{char}]{ naipe}
\end{DoxyParamCaption}
)}}
\label{descarte_8c_a7639a5e1a5d758731fe19ad2a258621f}
Função interna. É utilizada apenas em outras funções. 
\begin{DoxyParams}{Parâmetros}
\item[{\em D}]-\/ Baralho de descarte. \item[{\em carta}]-\/ número da carta que vai ser inserida. \item[{\em naipe}]-\/ naipe da carta que vai ser inserida. \end{DoxyParams}
\hypertarget{descarte_8c_ae868e6817465c707db38fee6a1ac0ad8}{
\index{descarte.c@{descarte.c}!mostraDescarte@{mostraDescarte}}
\index{mostraDescarte@{mostraDescarte}!descarte.c@{descarte.c}}
\subsubsection[{mostraDescarte}]{\setlength{\rightskip}{0pt plus 5cm}{\bf lista} mostraDescarte (
\begin{DoxyParamCaption}
\item[{{\bf Baralho}}]{ D, }
\item[{int}]{ pos}
\end{DoxyParamCaption}
)}}
\label{descarte_8c_ae868e6817465c707db38fee6a1ac0ad8}
Retorna a carta que está em uma determinada posição do baralho de descarte. 
\begin{DoxyParams}{Parâmetros}
\item[{\em D}]-\/ Baralho de descarte. \item[{\em pos}]-\/ Posição da carta. \end{DoxyParams}
\hypertarget{descarte_8c_a1145f049a5b9cbb1b61842eb0a3948ee}{
\index{descarte.c@{descarte.c}!retiraDescarte@{retiraDescarte}}
\index{retiraDescarte@{retiraDescarte}!descarte.c@{descarte.c}}
\subsubsection[{retiraDescarte}]{\setlength{\rightskip}{0pt plus 5cm}{\bf lista} retiraDescarte (
\begin{DoxyParamCaption}
\item[{{\bf Baralho} $\ast$}]{ D, }
\item[{int}]{ pos}
\end{DoxyParamCaption}
)}}
\label{descarte_8c_a1145f049a5b9cbb1b61842eb0a3948ee}
Remove uma carta do baralho de descarte e retorna a mesma. 
\begin{DoxyParams}{Parâmetros}
\item[{\em D}]-\/ Baralho de descarte. \item[{\em pos}]-\/ Posição do baralho em que vai ser removida a carta. \end{DoxyParams}

\hypertarget{descarte_8h}{
\section{Referência do Arquivo descarte.h}
\label{descarte_8h}\index{descarte.h@{descarte.h}}
}
\subsection*{Funções}
\begin{DoxyCompactItemize}
\item 
\hyperlink{structnodo}{Baralho} \hyperlink{descarte_8h_afb7bcec544cd6814429992faa58a4d35}{atualizaDescarte} (\hyperlink{structnodo}{Baralho} D)
\item 
\hyperlink{structnodo}{lista} \hyperlink{descarte_8h_ae868e6817465c707db38fee6a1ac0ad8}{mostraDescarte} (\hyperlink{structnodo}{Baralho} D, int pos)
\item 
\hyperlink{structnodo}{Baralho} \hyperlink{descarte_8h_a7639a5e1a5d758731fe19ad2a258621f}{insereCartaDescarte} (\hyperlink{structnodo}{Baralho} D, int carta, char naipe)
\item 
\hyperlink{structnodo}{lista} \hyperlink{descarte_8h_a1145f049a5b9cbb1b61842eb0a3948ee}{retiraDescarte} (\hyperlink{structnodo}{Baralho} $\ast$D, int pos)
\end{DoxyCompactItemize}


\subsection{Funções}
\hypertarget{descarte_8h_afb7bcec544cd6814429992faa58a4d35}{
\index{descarte.h@{descarte.h}!atualizaDescarte@{atualizaDescarte}}
\index{atualizaDescarte@{atualizaDescarte}!descarte.h@{descarte.h}}
\subsubsection[{atualizaDescarte}]{\setlength{\rightskip}{0pt plus 5cm}{\bf Baralho} atualizaDescarte (
\begin{DoxyParamCaption}
\item[{{\bf Baralho}}]{ D}
\end{DoxyParamCaption}
)}}
\label{descarte_8h_afb7bcec544cd6814429992faa58a4d35}
\begin{DoxyAuthor}{Autores}
Cristiano Ribeiro, Evair Severo \href{mailto:cristiano.daitx@gmail.com}{\tt cristiano.daitx@gmail.com}, \href{mailto:evairsevero@gmail.com}{\tt evairsevero@gmail.com}
\end{DoxyAuthor}
Insere uma determinada carta no monte de descarte. 
\begin{DoxyParams}{Parâmetros}
\item[{\em D}]-\/ Baralho de descarte onde vai ser inserida a carta. \end{DoxyParams}
\hypertarget{descarte_8h_a7639a5e1a5d758731fe19ad2a258621f}{
\index{descarte.h@{descarte.h}!insereCartaDescarte@{insereCartaDescarte}}
\index{insereCartaDescarte@{insereCartaDescarte}!descarte.h@{descarte.h}}
\subsubsection[{insereCartaDescarte}]{\setlength{\rightskip}{0pt plus 5cm}{\bf Baralho} insereCartaDescarte (
\begin{DoxyParamCaption}
\item[{{\bf Baralho}}]{ D, }
\item[{int}]{ carta, }
\item[{char}]{ naipe}
\end{DoxyParamCaption}
)}}
\label{descarte_8h_a7639a5e1a5d758731fe19ad2a258621f}
Função interna. É utilizada apenas em outras funções. 
\begin{DoxyParams}{Parâmetros}
\item[{\em D}]-\/ Baralho de descarte. \item[{\em carta}]-\/ número da carta que vai ser inserida. \item[{\em naipe}]-\/ naipe da carta que vai ser inserida. \end{DoxyParams}
\hypertarget{descarte_8h_ae868e6817465c707db38fee6a1ac0ad8}{
\index{descarte.h@{descarte.h}!mostraDescarte@{mostraDescarte}}
\index{mostraDescarte@{mostraDescarte}!descarte.h@{descarte.h}}
\subsubsection[{mostraDescarte}]{\setlength{\rightskip}{0pt plus 5cm}{\bf lista} mostraDescarte (
\begin{DoxyParamCaption}
\item[{{\bf Baralho}}]{ D, }
\item[{int}]{ pos}
\end{DoxyParamCaption}
)}}
\label{descarte_8h_ae868e6817465c707db38fee6a1ac0ad8}
Retorna a carta que está em uma determinada posição do baralho de descarte. 
\begin{DoxyParams}{Parâmetros}
\item[{\em D}]-\/ Baralho de descarte. \item[{\em pos}]-\/ Posição da carta. \end{DoxyParams}
\hypertarget{descarte_8h_a1145f049a5b9cbb1b61842eb0a3948ee}{
\index{descarte.h@{descarte.h}!retiraDescarte@{retiraDescarte}}
\index{retiraDescarte@{retiraDescarte}!descarte.h@{descarte.h}}
\subsubsection[{retiraDescarte}]{\setlength{\rightskip}{0pt plus 5cm}{\bf lista} retiraDescarte (
\begin{DoxyParamCaption}
\item[{{\bf Baralho} $\ast$}]{ D, }
\item[{int}]{ pos}
\end{DoxyParamCaption}
)}}
\label{descarte_8h_a1145f049a5b9cbb1b61842eb0a3948ee}
Remove uma carta do baralho de descarte e retorna a mesma. 
\begin{DoxyParams}{Parâmetros}
\item[{\em D}]-\/ Baralho de descarte. \item[{\em pos}]-\/ Posição do baralho em que vai ser removida a carta. \end{DoxyParams}

\hypertarget{embaralhar_8c}{
\section{Referência do Arquivo embaralhar.c}
\label{embaralhar_8c}\index{embaralhar.c@{embaralhar.c}}
}
{\ttfamily \#include $<$stdio.h$>$}\par
{\ttfamily \#include $<$stdlib.h$>$}\par
{\ttfamily \#include $<$malloc.h$>$}\par
{\ttfamily \#include $<$time.h$>$}\par
{\ttfamily \#include $<$string.h$>$}\par
{\ttfamily \#include \char`\"{}embaralhar.h\char`\"{}}\par
\subsection*{Funções}
\begin{DoxyCompactItemize}
\item 
int \hyperlink{embaralhar_8c_ade62c2f5f8a33a7eec005a5a89436f7c}{verificaCarta} (int carta, int naipe, \hyperlink{structnodo}{Baralho} B)
\item 
\hyperlink{structnodo}{Baralho} \hyperlink{embaralhar_8c_ad75326bbd785dfd82a0376620fa1290f}{insereCarta} (\hyperlink{structnodo}{Baralho} B, int carta, int naipe)
\item 
void \hyperlink{embaralhar_8c_a80deac0aee3c981fa8c7556505a96823}{imprime} (\hyperlink{structnodo}{Baralho} B)
\item 
\hyperlink{structnodo}{Baralho} \hyperlink{embaralhar_8c_a7063c0087949c844bcf617bb0fe9ec1a}{embaralhar} (\hyperlink{structnodo}{Baralho} B)
\item 
\hyperlink{structnodo}{Baralho} \hyperlink{embaralhar_8c_a6d427e3d4a2a4ca9b60e1e5251e7412d}{cria} (void)
\end{DoxyCompactItemize}


\subsection{Funções}
\hypertarget{embaralhar_8c_a6d427e3d4a2a4ca9b60e1e5251e7412d}{
\index{embaralhar.c@{embaralhar.c}!cria@{cria}}
\index{cria@{cria}!embaralhar.c@{embaralhar.c}}
\subsubsection[{cria}]{\setlength{\rightskip}{0pt plus 5cm}{\bf Baralho} cria (
\begin{DoxyParamCaption}
\item[{void}]{}
\end{DoxyParamCaption}
)}}
\label{embaralhar_8c_a6d427e3d4a2a4ca9b60e1e5251e7412d}
Cria um baralho como NULL. \hypertarget{embaralhar_8c_a7063c0087949c844bcf617bb0fe9ec1a}{
\index{embaralhar.c@{embaralhar.c}!embaralhar@{embaralhar}}
\index{embaralhar@{embaralhar}!embaralhar.c@{embaralhar.c}}
\subsubsection[{embaralhar}]{\setlength{\rightskip}{0pt plus 5cm}{\bf Baralho} embaralhar (
\begin{DoxyParamCaption}
\item[{{\bf Baralho}}]{ B}
\end{DoxyParamCaption}
)}}
\label{embaralhar_8c_a7063c0087949c844bcf617bb0fe9ec1a}
Embaralha um baralho de 52 cartas. 
\begin{DoxyParams}{Parâmetros}
\item[{\em B}]-\/ Baralho a ser embaralhado. \end{DoxyParams}
\hypertarget{embaralhar_8c_a80deac0aee3c981fa8c7556505a96823}{
\index{embaralhar.c@{embaralhar.c}!imprime@{imprime}}
\index{imprime@{imprime}!embaralhar.c@{embaralhar.c}}
\subsubsection[{imprime}]{\setlength{\rightskip}{0pt plus 5cm}void imprime (
\begin{DoxyParamCaption}
\item[{{\bf Baralho}}]{ B}
\end{DoxyParamCaption}
)}}
\label{embaralhar_8c_a80deac0aee3c981fa8c7556505a96823}
\hypertarget{embaralhar_8c_ad75326bbd785dfd82a0376620fa1290f}{
\index{embaralhar.c@{embaralhar.c}!insereCarta@{insereCarta}}
\index{insereCarta@{insereCarta}!embaralhar.c@{embaralhar.c}}
\subsubsection[{insereCarta}]{\setlength{\rightskip}{0pt plus 5cm}{\bf Baralho} insereCarta (
\begin{DoxyParamCaption}
\item[{{\bf Baralho}}]{ B, }
\item[{int}]{ carta, }
\item[{int}]{ naipe}
\end{DoxyParamCaption}
)}}
\label{embaralhar_8c_ad75326bbd785dfd82a0376620fa1290f}
\hypertarget{embaralhar_8c_ade62c2f5f8a33a7eec005a5a89436f7c}{
\index{embaralhar.c@{embaralhar.c}!verificaCarta@{verificaCarta}}
\index{verificaCarta@{verificaCarta}!embaralhar.c@{embaralhar.c}}
\subsubsection[{verificaCarta}]{\setlength{\rightskip}{0pt plus 5cm}int verificaCarta (
\begin{DoxyParamCaption}
\item[{int}]{ carta, }
\item[{int}]{ naipe, }
\item[{{\bf Baralho}}]{ B}
\end{DoxyParamCaption}
)}}
\label{embaralhar_8c_ade62c2f5f8a33a7eec005a5a89436f7c}

\hypertarget{embaralhar_8h}{
\section{Referência do Arquivo embaralhar.h}
\label{embaralhar_8h}\index{embaralhar.h@{embaralhar.h}}
}
\subsection*{Estruturas de Dados}
\begin{DoxyCompactItemize}
\item 
struct \hyperlink{structnodo}{nodo}
\end{DoxyCompactItemize}
\subsection*{Definições de Tipos}
\begin{DoxyCompactItemize}
\item 
typedef struct \hyperlink{structnodo}{nodo} \hyperlink{embaralhar_8h_a320a4f7845c53306e254a8ac96f4fad7}{lista}
\item 
typedef \hyperlink{structnodo}{lista} $\ast$ \hyperlink{embaralhar_8h_a55f1cef7a68d15235c49fa7ca22afed3}{Baralho}
\begin{DoxyCompactList}\small\item\em Lista que representa um baralho. \item\end{DoxyCompactList}\end{DoxyCompactItemize}
\subsection*{Funções}
\begin{DoxyCompactItemize}
\item 
\hyperlink{structnodo}{Baralho} \hyperlink{embaralhar_8h_a6d427e3d4a2a4ca9b60e1e5251e7412d}{cria} (void)
\item 
\hyperlink{structnodo}{Baralho} \hyperlink{embaralhar_8h_a7063c0087949c844bcf617bb0fe9ec1a}{embaralhar} (\hyperlink{structnodo}{Baralho} B)
\item 
int \hyperlink{embaralhar_8h_acac98e14e97aca23de0535d38b0961d1}{verifica\_\-Carta} (\hyperlink{structnodo}{Baralho} B, int carta, int naipe)
\item 
\hyperlink{structnodo}{Baralho} \hyperlink{embaralhar_8h_af278e53e84d11de6cb995569d3f74681}{insere\_\-Carta} (\hyperlink{structnodo}{Baralho} B, int carta, int naipe)
\item 
void \hyperlink{embaralhar_8h_a80deac0aee3c981fa8c7556505a96823}{imprime} (\hyperlink{structnodo}{Baralho} B)
\end{DoxyCompactItemize}


\subsection{Definições dos tipos}
\hypertarget{embaralhar_8h_a55f1cef7a68d15235c49fa7ca22afed3}{
\index{embaralhar.h@{embaralhar.h}!Baralho@{Baralho}}
\index{Baralho@{Baralho}!embaralhar.h@{embaralhar.h}}
\subsubsection[{Baralho}]{\setlength{\rightskip}{0pt plus 5cm}typedef {\bf lista}$\ast$ {\bf Baralho}}}
\label{embaralhar_8h_a55f1cef7a68d15235c49fa7ca22afed3}


Lista que representa um baralho. 

\hypertarget{embaralhar_8h_a320a4f7845c53306e254a8ac96f4fad7}{
\index{embaralhar.h@{embaralhar.h}!lista@{lista}}
\index{lista@{lista}!embaralhar.h@{embaralhar.h}}
\subsubsection[{lista}]{\setlength{\rightskip}{0pt plus 5cm}typedef struct {\bf nodo} {\bf lista}}}
\label{embaralhar_8h_a320a4f7845c53306e254a8ac96f4fad7}
Estrutura que representa uma carta. Armazena a carta e seu respectivo naipe. Sendo as cartas representadas de 1 a 13. Logo as cartas A, J, Q e K são representadas, pelos números 1, 11, 12 e 13 respectivamente. As demais cartas seguem o padrão que conhecemos. Os nipes são representados por um caractere. Paus, copas, espadas e ouro são representados por P, C, E e O respectivamente. 

\subsection{Funções}
\hypertarget{embaralhar_8h_a6d427e3d4a2a4ca9b60e1e5251e7412d}{
\index{embaralhar.h@{embaralhar.h}!cria@{cria}}
\index{cria@{cria}!embaralhar.h@{embaralhar.h}}
\subsubsection[{cria}]{\setlength{\rightskip}{0pt plus 5cm}{\bf Baralho} cria (
\begin{DoxyParamCaption}
\item[{void}]{}
\end{DoxyParamCaption}
)}}
\label{embaralhar_8h_a6d427e3d4a2a4ca9b60e1e5251e7412d}
Cria um baralho como NULL. \hypertarget{embaralhar_8h_a7063c0087949c844bcf617bb0fe9ec1a}{
\index{embaralhar.h@{embaralhar.h}!embaralhar@{embaralhar}}
\index{embaralhar@{embaralhar}!embaralhar.h@{embaralhar.h}}
\subsubsection[{embaralhar}]{\setlength{\rightskip}{0pt plus 5cm}{\bf Baralho} embaralhar (
\begin{DoxyParamCaption}
\item[{{\bf Baralho}}]{ B}
\end{DoxyParamCaption}
)}}
\label{embaralhar_8h_a7063c0087949c844bcf617bb0fe9ec1a}
Embaralha um baralho de 52 cartas. 
\begin{DoxyParams}{Parâmetros}
\item[{\em B}]-\/ Baralho a ser embaralhado. \end{DoxyParams}
\hypertarget{embaralhar_8h_a80deac0aee3c981fa8c7556505a96823}{
\index{embaralhar.h@{embaralhar.h}!imprime@{imprime}}
\index{imprime@{imprime}!embaralhar.h@{embaralhar.h}}
\subsubsection[{imprime}]{\setlength{\rightskip}{0pt plus 5cm}void imprime (
\begin{DoxyParamCaption}
\item[{{\bf Baralho}}]{ B}
\end{DoxyParamCaption}
)}}
\label{embaralhar_8h_a80deac0aee3c981fa8c7556505a96823}
\hypertarget{embaralhar_8h_af278e53e84d11de6cb995569d3f74681}{
\index{embaralhar.h@{embaralhar.h}!insere\_\-Carta@{insere\_\-Carta}}
\index{insere\_\-Carta@{insere\_\-Carta}!embaralhar.h@{embaralhar.h}}
\subsubsection[{insere\_\-Carta}]{\setlength{\rightskip}{0pt plus 5cm}{\bf Baralho} insere\_\-Carta (
\begin{DoxyParamCaption}
\item[{{\bf Baralho}}]{ B, }
\item[{int}]{ carta, }
\item[{int}]{ naipe}
\end{DoxyParamCaption}
)}}
\label{embaralhar_8h_af278e53e84d11de6cb995569d3f74681}
Insere uma determinada carta em um baralho. 
\begin{DoxyParams}{Parâmetros}
\item[{\em B}]-\/ Baralho onde a carta vai ser inserida. \item[{\em carta}]-\/ Número da carta a ser inserida. \item[{\em naipe}]-\/ Naipe da carta. \end{DoxyParams}
\hypertarget{embaralhar_8h_acac98e14e97aca23de0535d38b0961d1}{
\index{embaralhar.h@{embaralhar.h}!verifica\_\-Carta@{verifica\_\-Carta}}
\index{verifica\_\-Carta@{verifica\_\-Carta}!embaralhar.h@{embaralhar.h}}
\subsubsection[{verifica\_\-Carta}]{\setlength{\rightskip}{0pt plus 5cm}int verifica\_\-Carta (
\begin{DoxyParamCaption}
\item[{{\bf Baralho}}]{ B, }
\item[{int}]{ carta, }
\item[{int}]{ naipe}
\end{DoxyParamCaption}
)}}
\label{embaralhar_8h_acac98e14e97aca23de0535d38b0961d1}
Verifica se uma determinada carta está no baralho. 
\begin{DoxyParams}{Parâmetros}
\item[{\em B}]-\/ Baralho onde a carta vai ser pesquisada. \item[{\em carta}]-\/ Número da carta a ser pesquisada. \item[{\em naipe}]-\/ Naipe da carta. \end{DoxyParams}

\hypertarget{retira_8c}{
\section{Referência do Arquivo retira.c}
\label{retira_8c}\index{retira.c@{retira.c}}
}
{\ttfamily \#include $<$stdio.h$>$}\par
{\ttfamily \#include $<$stdlib.h$>$}\par
{\ttfamily \#include $<$malloc.h$>$}\par
{\ttfamily \#include $<$time.h$>$}\par
{\ttfamily \#include $<$string.h$>$}\par
{\ttfamily \#include \char`\"{}embaralhar.h\char`\"{}}\par
{\ttfamily \#include \char`\"{}retira.h\char`\"{}}\par
{\ttfamily \#include \char`\"{}corta.h\char`\"{}}\par
\subsection*{Funções}
\begin{DoxyCompactItemize}
\item 
\hyperlink{structnodo}{lista} \hyperlink{retira_8c_ad7c06faa5b9ab3b65bf3a2e373bd66e1}{retira\_\-inicio} (\hyperlink{structnodo}{Baralho} $\ast$B)
\item 
\hyperlink{structnodo}{lista} \hyperlink{retira_8c_aa01c3e7f46911b77dd4f1ac4978f78c6}{retira\_\-final} (\hyperlink{structnodo}{Baralho} $\ast$B)
\end{DoxyCompactItemize}


\subsection{Funções}
\hypertarget{retira_8c_aa01c3e7f46911b77dd4f1ac4978f78c6}{
\index{retira.c@{retira.c}!retira\_\-final@{retira\_\-final}}
\index{retira\_\-final@{retira\_\-final}!retira.c@{retira.c}}
\subsubsection[{retira\_\-final}]{\setlength{\rightskip}{0pt plus 5cm}{\bf lista} retira\_\-final (
\begin{DoxyParamCaption}
\item[{{\bf Baralho} $\ast$}]{ B}
\end{DoxyParamCaption}
)}}
\label{retira_8c_aa01c3e7f46911b77dd4f1ac4978f78c6}
Retira a carta que está no final do baralho. O retorno dessa função será um nodo com essa carta. 
\begin{DoxyParams}{Parâmetros}
\item[{\em B}]-\/ Baralho a ser retirada a carta. \end{DoxyParams}
\hypertarget{retira_8c_ad7c06faa5b9ab3b65bf3a2e373bd66e1}{
\index{retira.c@{retira.c}!retira\_\-inicio@{retira\_\-inicio}}
\index{retira\_\-inicio@{retira\_\-inicio}!retira.c@{retira.c}}
\subsubsection[{retira\_\-inicio}]{\setlength{\rightskip}{0pt plus 5cm}{\bf lista} retira\_\-inicio (
\begin{DoxyParamCaption}
\item[{{\bf Baralho} $\ast$}]{ B}
\end{DoxyParamCaption}
)}}
\label{retira_8c_ad7c06faa5b9ab3b65bf3a2e373bd66e1}
Retira a carta que está no inicio do baralho. O retorno dessa função será um nodo com essa carta. 
\begin{DoxyParams}{Parâmetros}
\item[{\em B}]-\/ Baralho a ser retirada a carta. \end{DoxyParams}

\hypertarget{retira_8h}{
\section{Referência do Arquivo retira.h}
\label{retira_8h}\index{retira.h@{retira.h}}
}
\subsection*{Funções}
\begin{DoxyCompactItemize}
\item 
\hyperlink{structnodo}{lista} \hyperlink{retira_8h_ad7c06faa5b9ab3b65bf3a2e373bd66e1}{retira\_\-inicio} (\hyperlink{structnodo}{Baralho} $\ast$B)
\item 
\hyperlink{structnodo}{lista} \hyperlink{retira_8h_aa01c3e7f46911b77dd4f1ac4978f78c6}{retira\_\-final} (\hyperlink{structnodo}{Baralho} $\ast$B)
\end{DoxyCompactItemize}


\subsection{Funções}
\hypertarget{retira_8h_aa01c3e7f46911b77dd4f1ac4978f78c6}{
\index{retira.h@{retira.h}!retira\_\-final@{retira\_\-final}}
\index{retira\_\-final@{retira\_\-final}!retira.h@{retira.h}}
\subsubsection[{retira\_\-final}]{\setlength{\rightskip}{0pt plus 5cm}{\bf lista} retira\_\-final (
\begin{DoxyParamCaption}
\item[{{\bf Baralho} $\ast$}]{ B}
\end{DoxyParamCaption}
)}}
\label{retira_8h_aa01c3e7f46911b77dd4f1ac4978f78c6}
Retira a carta que está no final do baralho. O retorno dessa função será um nodo com essa carta. 
\begin{DoxyParams}{Parâmetros}
\item[{\em B}]-\/ Baralho a ser retirada a carta. \end{DoxyParams}
\hypertarget{retira_8h_ad7c06faa5b9ab3b65bf3a2e373bd66e1}{
\index{retira.h@{retira.h}!retira\_\-inicio@{retira\_\-inicio}}
\index{retira\_\-inicio@{retira\_\-inicio}!retira.h@{retira.h}}
\subsubsection[{retira\_\-inicio}]{\setlength{\rightskip}{0pt plus 5cm}{\bf lista} retira\_\-inicio (
\begin{DoxyParamCaption}
\item[{{\bf Baralho} $\ast$}]{ B}
\end{DoxyParamCaption}
)}}
\label{retira_8h_ad7c06faa5b9ab3b65bf3a2e373bd66e1}
Retira a carta que está no inicio do baralho. O retorno dessa função será um nodo com essa carta. 
\begin{DoxyParams}{Parâmetros}
\item[{\em B}]-\/ Baralho a ser retirada a carta. \end{DoxyParams}

\printindex
\end{document}
